\section{Problem 6}

\subsection{Instruction}

Download the gzipped archive gridDataUt1.tar.gz from
http://radionavlab.ae.utexas.edu/datastore/gnssSigProcCourse/. Repeat the steps
in problem 4 for this data set. Do this for TXID 29 and for TXID 31. Note that
the data capture interval is much shorter, and the ionospheric delay changes
much less significantly, than in the original data set for problem 4.

The following broadcast ionospheric parameters are valid for the data in
gridDataUt1.tar.gz:

\begin{verbatim}
alpha0: 4.6566e-009
alpha1: 1.4901e-008
alpha2: -5.9605e-008
alpha3: -5.9605e-008
beta0: 79872
beta1: 65536
beta2: -65536
beta3: -393220
\end{verbatim}

Assume a true GPS time of receipt given by

\begin{verbatim}
week = 1490
seconds = 146238.774036515
\end{verbatim}

and a static ECEF receiver antenna location given in meters by

\begin{verbatim}
X = 1101972.5309609
Y = -4583489.78279095
Z = 4282244.3010423
\end{verbatim}

Use the function you wrote for problem 5 to determine the ionospheric delay as
calculated by the broadcast model for TXIDs 29 and 31, with the ECEF SV position
of TXID 29 at time of transmission given in meters by

\begin{verbatim}
X = 24597807.6872883
Y = -3065999.1384585
Z = 9611346.77939927
\end{verbatim}

and the ECEF SV position of TXID 31 at time of transmission given in meters by

\begin{verbatim}
X = 2339172.27088689
Y = -16191391.3551878
Z = 21104185.0481546
\end{verbatim}

Compare the ionospheric delay as calculated by the broadcast model for TXIDs 29
and 31 with your corresponding plots for these two. How much difference is there
between the model-calculated and the empirical ionospheric delays? To what do you
attribute this discrepancy?

\subsection{Solution}


