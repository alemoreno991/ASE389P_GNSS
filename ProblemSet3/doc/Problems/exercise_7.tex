\section{Problem 7}

\subsection{Instruction}

In lecture, we noted that the phase shift across an ionospheric blob
(irregularity) of size a (in meters) in excess of the phase shift across a layer
of average background density of the same size is
$\Delta \phi_0 = a r_e \lambda \Delta\eta_e$ where $r_e$ is the classical
electron radius in meters, $\lambda$ is the wavelength of the carrier in meters,
and $\Delta \eta_e$ is the excess electron density within the blob as compared
to the average background density, in electrons per cubic meter. Derive this
expression from expressions for the refractive index $\eta$ and the excess phase
delay $\Delta \tau_p$ given in lecture.

\subsection{Solution}

The excess delay is defined as follows

\begin{equation}
	\Delta \tau_p = \frac{1}{c} \int_{SV}^{RX} \left[ \eta(l) - 1 \right] dl
	\label{eq:excess_delay}
\end{equation}

It is possible to think of scintillation as some electron density anomaly (blob)
in the path of the signal. Therefore, it is possible to write something like

\begin{equation*}
	\Delta \tau_p = \frac{1}{c} \int_{SV}^{RX} \left[ \eta_{avg}(l) - 1 \right] dl
	+ \frac{1}{c} \int_{blob} \left[ \eta_{blob}(l) - 1 \right] dl
\end{equation*}


\begin{eqnarray*}
	\Delta \tilde{\tau}_p &=& \frac{1}{c} \int_{blob} \left[ \eta_{blob}(l) - 1 \right] dl \\
	&=& \frac{1}{c} \int_{0}^{a} \left[ (1 - \frac{r_e \Delta \eta_e \lambda^2}{2 \pi} ) - 1 \right] dl \\
	&=& - \frac{r_e \Delta \eta_e \lambda^2}{2 \pi c}  \int_{0}^{a} 1 dl \\
	&=& - \frac{a r_e \Delta \eta_e \lambda^2}{2 \pi c}
\end{eqnarray*}

Recognizing that $\Delta \tilde{\tau}_p$ is the excess delay in seconds associated
with the scintillation effect of the blob. Then it is possible to compute the
phase shift as follows. First, $c \Delta \tilde{\tau}_p$ is the delay in meters.
Then, $\frac{c \Delta \tilde{\tau}_p}{\lambda}$ is the delay expressed as
a fraction of wave length. Finally the phase shift can be obtained through

\begin{equation*}
	\Delta \phi_0 = \frac{2 \pi c \Delta \tilde{\tau}_p}{\lambda}
	= - a r_e \Delta \eta_e \lambda
\end{equation*}

