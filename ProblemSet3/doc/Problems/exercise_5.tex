\section{Problem 5}

\subsection{Instruction}

Write a function in Matlab for computing the ionospheric delay from a model of
the ionosphere. Your function should adhere to the interface described on the
next page (which you can copy and paste as comments to your function). Only
develop calculations for the broadcast (Klobuchar) model. The function can later
be augmented to accommodate other model types. You can learn about the broadcast
model on pages 168-169 of the Misra and Enge text. More details can be found on
pages 128-130 of the GPS interface specification (IS) IS-GPS-200F.pdf posted on
Canvas. You will also need to write your own function for computing satellite
elevation and azimuth angles. Assume the WGS84 model for the shape of the Earth.
Note that the GPS IS uses semicircles as its angular measure, which is a strange
convention employed back in the 1970s to reduce memory and computation
requirement. The sin and cos functions in the IS (e.g., in Fig. 20-4) are meant
to operate on semicircles, unless otherwise indicated. Thus, when the IS writes,
e.g., $cos(\lambda_i −1.1617)$, where $\lambda_i$ is given in semicircles, you
can implement this in Matlab as $cos((\lambda_i −1.1617)\pi)$.

\subsection{Solution}


\begin{lstlisting}
function [delTauG] = getIonoDelay(ionodata,fc,rRx,rSv,tGPS,model)
% getIonoDelay : Return a model-based estimate of the ionospheric delay
%                experienced by a trans-ionospheric GNSS signal as it
%                propagates from a GNSS SV to the antenna of a terrestrial
%                GNSS receiver.
%
% INPUTS
%
% ionodata ------- Structure containing a parameterization of the
%                  ionosphere that is valid at time tGPS. The structure is
%                  defined differently depending on what ionospheric model
%                  is selected:
%
%                  broadcast --- For the broadcast (Klobuchar) model, ionodata
%                                is a structure containing the following fields:
%
%                       alpha0 ... alpha3 -- power series expansion coefficients
%                                            for amplitude of ionospheric delay
%                       beta0 ... beta3 -- power series expansion coefficients
%                                          for period of ionospheric plasma density 
%                                          cycle
%
%
% Other models TBD ...
%
% fc ------------- Carrier frequency of the GNSS signal, in Hz.
%
% rRx ------------ A 3-by-1 vector representing the receiver antenna position
%                  at the time of receipt of the signal, expressed in meters
%                  in the ECEF reference frame.
%
% rSv ------------ A 3-by-1 vector representing the space vehicle antenna
%                  position at the time of transmission of the signal,
%                  expressed in meters in the ECEF reference frame.
%
% tGPS ----------- A structure containing the true GPS time of receipt of
%                  the signal. The structure has the following fields:
%                  week -- unambiguous GPS week number
%                  seconds -- seconds (including fractional seconds) of the
%                  GPS week
%
% model ---------- A string identifying the model to be used in the
%                  computation of the ionospheric delay:
%                  broadcast --- The broadcast (Klobuchar) model.
%
% Other models TBD ...
%
% OUTPUTS
%
% delTauG -------- Modeled scalar excess group ionospheric delay experienced
%                  by the transionospheric GNSS signal, in seconds.
%
%+----------------------------------------------------------------------------+
% References: For the broadcast (Klobuchar) model, see IS-GPS-200F
% pp. 128-130.
%
%+============================================================================+
wgs84 = wgs84Ellipsoid('meter');
[lat,lon,h] = ecef2geodetic(wgs84, rRx(1), rRx(2), rRx(3));
[az,elev,slantRange] = ecef2aer(rSv(1), rSv(2), rSv(3), lat, lon, h, wgs84);
lambda_u = lon/180; % user geodetic longitude (semi-circles)
phi_u    = lat/180; % user geodetic latitude (semi-circles) 
A        = az/180;  % azimuth angle between user and satellite, measured clockwise 
                % positive from the true North (semi-circles) 
E        = elev/180;% elevation angle between user and satellite (semi_circle)

% earth's  central  angle  between  the  user  position  and  the  earth  
% projection  of ionospheric intersection point (semi-circles) 
Psi = 0.0137/(E+0.11) - 0.022;

% geodetic  latitude  of  the  earth  projection  of  the  ionospheric  
% intersection  point (semi-circles) 
phi_i = phi_u + Psi * cos(A*pi);
phi_i = max(-0.416, phi_i);
phi_i = min(0.416, phi_i);

% geodetic  longitude  of  the  earth  projection  of  the  ionospheric  
% intersection  point (semi-circles) 
lambda_i = lambda_u + Psi*cos(A*pi)/cos(phi_i*pi);

% geomagnetic latitude of the earth projection of the ionospheric 
% intersection point (mean ionospheric height assumed 350 km) (semi-circles) 
phi_m = phi_i + 0.064*cos((lambda_i - 1.617)*pi);

% local time (sec) 
t = 4.32*(10^4)*lambda_i + tGPS.seconds;
while t > 86400
    t = t - 86400;
end
while t < 0
    t = t + 86400;
end

PER = ionodata.broadcast.beta0 * phi_m^0 + ...
      ionodata.broadcast.beta1 * phi_m^1 + ...
      ionodata.broadcast.beta2 * phi_m^2 + ...
      ionodata.broadcast.beta3 * phi_m^3;
PER = max(72000, PER);


AMP = ionodata.broadcast.alpha0 * phi_m^0 + ...
      ionodata.broadcast.alpha1 * phi_m^1 + ...
      ionodata.broadcast.alpha2 * phi_m^2 + ...
      ionodata.broadcast.alpha3 * phi_m^3;
AMP = max(0, AMP);

% phase (radians) 
x = 2*pi*(t - 50400)/PER;

% obliquity factor (dimensionless) 
F = 1 + 16*(0.53 - E)^3;

% Estimate the ionospheric delay
if abs(x) < 1.57
    T_iono = F*(5e-9 + AMP * (1 - (x^2)/2 + (x^4)/24));
else
    T_iono = F * 5e-9;
end

if fc == 1227.44 * 1e6
    gamma = (77/60)^2;
    T_iono = gamma * T_iono;
end

delTauG = T_iono;
end
\end{lstlisting}

\subsection{Results}

The solution I got for the conditions provided was $delTauG = 23.3269 ns$, which
translates to about 7 meters.
