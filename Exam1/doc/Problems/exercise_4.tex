\section{Problem 4}

\subsection{Instruction}

To measure the receiver temperature $T_R$ of a GNSS receiver and antenna setup,
a friend recommends placing the receiver’s antenna in a RF test enclosure such
as the one in the Radionavigation Laboratory (seen here
https://ramseytest.com/forensic-test-enclosures),
but cryogenically cooled down to 5 K, and then measuring the noise power in the
raw samples generated by the receiver. The enclosure effectively isolates the
antenna from environmental noise. The antenna is an active antenna consisting of
a patch element, a (passive) filter, and an amplifier. Is this a valid approach
for measuring TR ? Why or why not?

\subsection{Result}

The cryogenically cooled enclosure would only allow us to reduce the noise of
the receiver but not $T_A$. Therefore one could measure the noise power with and
without the cryogenically cooled enclosure and get an estimate of the $T_R$ by
substracting the results.

It seems like this is not a good idea though because if one would like to make the
best use of their money for the general usage of the GNSS receiver. Then, buying
the cryogenically cooled enclosure would only buy you a couple dBs better noise
performance. The main source of noise (the other GNSS signals one is not
interested in when aquiring from a particular satellite) would not be attenuated.
Therefore, one could measure the desired magnitude but that's it. No noticible
benefit would come from this purchase.


