\section{Problem 6}

\subsection{Instruction}

If C(t) is a random binary (± 1) spreading code with rectangular pulses, then,
as we showed in lecture, the power spectral density of C(t) is given by

\begin{equation}
	S_C (f) = T_C sinc^2 (f T_C )
\end{equation}

where $T_C$ is the chipping interval.

Suppose instead that C(t) has psd

\begin{equation}
	S_C (f) = T_C \Pi(f T_C )
\end{equation}

where $\Pi(f)$ is the rect function introduced in lecture. Imagine a
constellation of GNSS satellites broadcasting signals having this new $S_C (f)$.
Assume the signals have no data bit modulation, so that the signal’s baseband
representation is of the form

\begin{equation}
	r(t) = \sqrt{P} C(t) \exp[j \theta(t)]
\end{equation}

For this case, calculate $I_0 = S_I(0)$ for a single multiple-access interference
signal with power $P_I$, where $I_0$ and $S_I (f)$, the power spectrum of $I(t)$,
were defined in lecture. Now consider M multiple-access signals (including the
desired signal). If each of the M received signals has power $P_I$, at what value
of M does $I_0$ exceed $N_0$ ? Express your answer in terms of $N_0$, $T_C$ and
$P_I$.

\subsection{MATLAB code}

\begin{lstlisting}
%% Problem 4
% Suppose that C(t) has a power spectral density given by:
%
%                   Sc(f) = Tc * rectangularPulse(f*Tc)
%
% Where 'rectangularPulse' is the rect function introduced in lecture.
% Imagine a constellation of GNSS satellites broadcasting signals having
% this new Sc(f). Assume the signals have no data bit modulation, so that
% the signal's baseband representation is of the form:
%
%                   r(t) = sqrt(P)*C(t)*exp[j theta(t)]
%
clc, close all, clear all

% For this case, calculate I0 = SI(f=0) for a single multiple-access
% interference signal with power PI, where I0 and SI(f), the power spectrum
% of I(t), were defined in class.
%
% ANSWER:
% Following the same reasoning as in lecture:
% If CI(t) is a random-binary sequence "like C(t)" (same Tc), and if the
% two are uncorrelated. Then, (assuming fD=0)
%                       
%                   SrI(f) = PI*Sc(f)
% Thus, 
%                   SI(f) = Sc(f) x SrI(f) = Sc(f) x (PI*Sc(f))
%
%                   SI(f) = PI * integral_{-inf}^{inf}(Sc(f-x)Sc(x) dx
%
% Since, Sc(f) = Sc(-f)
%
%                   SI(f) = PI * integral_{-inf}^{inf}(Sc(x-f)Sc(x) dx
%
% Now, using the fact that the receiver has a low pass filter, only the f=0
% part of the spectral density leaks through. Therefore, it's ok to assume 
% I0 = SI(f=0) (REMEMBER: I0 is also a power density)
%
%     SI(0) = PI * integral_{-inf}^{inf}(Sc(x)^2) dx
%
%     SI(0) = PI * integral_{-inf}^{inf}(Tc*rectangularPulse(x*Tc)^2) dx
%
%     SI(0) = PI*Tc
I0 = PI*Tc;
I0_dBW_Hz = 10*log10(I0)

% Now consider M multiple-access signals (including the desired signal). If
% each of the M received signals has power PI, at what value of M does I0
% exceed N0? Express your answer in terms of N0, Tc, PI.
%
% ANSWER:
% The multiple-access interference is characterized by:
%
%           I0 = PI*Tc*(M-1)
%
% We want to know what M would result in I0 >= N0
%
%           PI*Tc*(M-1) >= N0
%           M >= N0/(PI*Tc) + 1, where M is integer
%        
% => I0 exceeds N0 when:
%           M = ceil(N0/(PI*Tc) + 1)
%
\end{lstlisting}
